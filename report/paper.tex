\documentclass[conference,10pt,letter]{IEEEtran}

\usepackage{url}
\usepackage{amssymb,amsthm}
\usepackage{graphicx,color}

%\usepackage{balance}

\usepackage{cite}
\usepackage{amsmath}
\usepackage{amssymb}

\usepackage{color, colortbl}
\usepackage{times}
\usepackage{caption}
\usepackage{rotating}
\usepackage{subcaption}

\newtheorem{theorem}{Theorem}
\newtheorem{example}{Example}
\newtheorem{definition}{Definition}
\newtheorem{lemma}{Lemma}

\newcommand{\XXXnote}[1]{{\bf\color{red} XXX: #1}}
\newcommand{\YYYnote}[1]{{\bf\color{red} YYY: #1}}
\newcommand*{\etal}{{\it et al.}}

\newcommand{\eat}[1]{}
\newcommand{\bi}{\begin{itemize}}
\newcommand{\ei}{\end{itemize}}
\newcommand{\im}{\item}
\newcommand{\eg}{{\it e.g.}\xspace}
\newcommand{\ie}{{\it i.e.}\xspace}
\newcommand{\etc}{{\it etc.}\xspace}

\def\P{\mathop{\mathsf{P}}}
\def\E{\mathop{\mathsf{E}}}

\begin{document}
\sloppy
\title{SOHO VPN box: Tunneling SOHO VPN traffic over TLS}
\maketitle
\begin{abstract}

In some countries network operators employ deep packet inspection techniques 
in order to block certain types of traffic. For example, Virtual Private Network 
(VPN) traffic can be analyzed and blocked to prevent non-cooperative users from 
sending encrypted packets over such networks. 

By observing that HTTPS works all over the world (configured for extremely large
number of web-servers) and cannot be easily analyzed (the payload is usually encrypted),
we argue that in the same manner VPN tunneling can be organized: By masquerading 
the VPN traffic with TLS or its older version - SSL, we can build a reliable and 
secure network. Packets, which are sent over such tunnels, can cross multiple
domains, which have various (strict and not so strict) security policies. 
Despite that the SSH can be potentially used to build such network, 
we have evidence that in certain countries connections
made over such tunnels are analyzed statistically: If the network utilization by 
such tunnels is high, bursts do exist, or connections are long living, then 
underlying TCP connections are reset by network operators. 

Thus, here we make an experimental effort in this direction: 
First, we describe different VPN solutions, which exist in the Internet; 
and, finally, we describe our experimental effort with Python based software,
which allows users to create VPN tunnel using TLS protocol. 

Unlike our previous effort here our goal is to route all traffic from small
office/home office through TLS tunnel.

\end{abstract}
\input intro.tex
\input background.tex
\input hardware.tex
\input experimental.tex
\input conclusions.tex
%\balance
\bibliographystyle{abbrv}
\bibliography{mybib}

\end{document}

